\documentclass{article}
\usepackage[utf8]{inputenc}
\usepackage[margin=0.70in]{geometry}
\usepackage{hyperref}
\usepackage{graphicx}
\usepackage{authblk}
\usepackage{subfig}
\usepackage{indentfirst}
\usepackage{minted}

\title{Program Assignment 3A: Load Bypass}
\author{Brian Park}
\affil{North Carolina State University, Computer Engineering 786}
\date{March 2023}

\begin{document}

\maketitle

\section{Load Bypass Implementation}
I implemented the load bypass by modifying some of the code already existing in \verb|ldst_unit|, specifically, \verb|ldst_unit::cycle()| and \verb|ldst_unit::memory_cycle()|. You can review the changes at the bottom of this report. It seems that \verb|ldst_unit::cycle()| is where memory accesses are generated and \verb|ldst_unit::memory_cycle()| is where memory unit actually executes the memory instructions, as indicated by the hints. In both functions, I modified the conditions to set the boolean flag \verb|bypassL1D| to true if the address of the memory request lies between \verb|0xc0000000-0xc00fffff|.

%For the correct implementation, you should be able to see that the L1 data cache access statistics are changed.

The results can be reproduced as follows (with \verb|SM6_TITANX| configuration), where each line represents the bypasses cache instructions per kernel:
\begin{verbatim}
cd ~/ISPASS/BFS
./ispass-2009-BFS graph65536.txt > gpuout.txt
cat gpuout.txt | grep "bypassed load instructions: "

bypassed load instructions: 52
bypassed load instructions: 326
bypassed load instructions: 1926
bypassed load instructions: 11769
bypassed load instructions: 68266
bypassed load instructions: 337901
bypassed load instructions: 965138
bypassed load instructions: 1090953
bypassed load instructions: 1095823
bypassed load instructions: 1095853
\end{verbatim}

For a correct implementation, it was hinted that the L1 data cache accesses statistics would change. We see that is the case. Because we let load instructions in the address range of \verb|0xc0000000-0xc00fffff| bypass the L1 data cache, intuitively there should be less accesses, which also seems to be the case. I noticed that BFS also has some accesses above \verb|0xc00fffff| when debugging, so that also confirms the difference.

Before:
\begin{verbatim}
cat gpuout_before.txt | grep "L1D_total_cache_accesses"
>>>
L1D_total_cache_accesses = 2116
L1D_total_cache_accesses = 4596
L1D_total_cache_accesses = 9307
L1D_total_cache_accesses = 27763
L1D_total_cache_accesses = 120908
L1D_total_cache_accesses = 508961
L1D_total_cache_accesses = 1230577
L1D_total_cache_accesses = 1478462
L1D_total_cache_accesses = 1488124
L1D_total_cache_accesses = 1490210
\end{verbatim}

After:
\begin{verbatim}
cat gpuout_after.txt | grep "L1D_total_cache_accesses"
>>>
L1D_total_cache_accesses = 2090
L1D_total_cache_accesses = 4433
L1D_total_cache_accesses = 8344
L1D_total_cache_accesses = 21879
L1D_total_cache_accesses = 86800
L1D_total_cache_accesses = 353055
L1D_total_cache_accesses = 875477
L1D_total_cache_accesses = 1060607
L1D_total_cache_accesses = 1067834
L1D_total_cache_accesses = 1069905
\end{verbatim}

How does this affect the miss rate? Shown below are the statistics before and after. 

Before:
\begin{verbatim}
cat gpuout_before.txt | grep "L1D_total_cache_miss_rate"
>>>
L1D_total_cache_miss_rate = 0.9849
L1D_total_cache_miss_rate = 0.9508
L1D_total_cache_miss_rate = 0.8465
L1D_total_cache_miss_rate = 0.6784
L1D_total_cache_miss_rate = 0.5801
L1D_total_cache_miss_rate = 0.5364
L1D_total_cache_miss_rate = 0.4957
L1D_total_cache_miss_rate = 0.5018
L1D_total_cache_miss_rate = 0.5032
L1D_total_cache_miss_rate = 0.5039
\end{verbatim}

After:
\begin{verbatim}
cat gpuout_after.txt | grep "L1D_total_cache_miss_rate"
>>>
L1D_total_cache_miss_rate = 0.9962
L1D_total_cache_miss_rate = 0.9824
L1D_total_cache_miss_rate = 0.9335
L1D_total_cache_miss_rate = 0.8364
L1D_total_cache_miss_rate = 0.7701
L1D_total_cache_miss_rate = 0.7261
L1D_total_cache_miss_rate = 0.6438
L1D_total_cache_miss_rate = 0.6353
L1D_total_cache_miss_rate = 0.6363
L1D_total_cache_miss_rate = 0.6370
\end{verbatim}


% 1. Write report to describe your implementation and report the counters about bypassed load instructions for BFS benchmark.
% 2. Zip the files you modified and your report as unity id.zip. Please make sure you follow the submission requirements correctly.

\newpage
For reference, this is the full diff of the code that was changed:
\begin{verbatim}
diff --git a/src/gpgpu-sim/shader.cc b/src/gpgpu-sim/shader.cc
index c6e7b8f..7e0bbbc 100644
--- a/src/gpgpu-sim/shader.cc
+++ b/src/gpgpu-sim/shader.cc
@@ -651,6 +651,9 @@ void shader_core_stats::print(FILE *fout) const {
               gpu_stall_shd_mem_breakdown[L_MEM_ST]
                                          [DATA_PORT_STALL]);  // data port stall
                                                               // at data cache
+
+
+  fprintf(fout, "bypassed load instructions: %d\n", gpgpu_n_bypassL1D);
   // fprintf(fout, "gpgpu_stall_shd_mem[g_mem_ld][mshr_rc] = %d\n",
   // gpu_stall_shd_mem_breakdown[G_MEM_LD][MSHR_RC_FAIL]); fprintf(fout,
   // "gpgpu_stall_shd_mem[g_mem_ld][icnt_rc] = %d\n",
@@ -1807,6 +1810,7 @@ mem_stage_stall_type ldst_unit::process_cache_access(
   mem_stage_stall_type result = NO_RC_FAIL;
   bool write_sent = was_write_sent(events);
   bool read_sent = was_read_sent(events);
+
   if (write_sent) {
     unsigned inc_ack = (m_config->m_L1D_config.get_mshr_type() == SECTOR_ASSOC)
                            ? (mf->get_data_size() / SECTOR_SIZE)
@@ -2039,6 +2043,7 @@ bool ldst_unit::memory_cycle(warp_inst_t &inst,
   const mem_access_t &access = inst.accessq_back();
 
   bool bypassL1D = false;
+
   if (CACHE_GLOBAL == inst.cache_op || (m_L1D == NULL)) {
     bypassL1D = true;
   } else if (inst.space.is_global()) {  // global memory access
@@ -2046,6 +2051,11 @@ bool ldst_unit::memory_cycle(warp_inst_t &inst,
     if (m_core->get_config()->gmem_skip_L1D && (CACHE_L1 != inst.cache_op))
       bypassL1D = true;
   }
+  if ((access.get_addr() >= 0xc0000000) && (access.get_addr() <= 0xc00fffff)) {
+      bypassL1D = true;
+      m_stats->gpgpu_n_bypassL1D++;
+  }
+
   if (bypassL1D) {
     // bypass L1 cache
     unsigned control_size =
@@ -2584,8 +2594,9 @@ void ldst_unit::cycle() {
                                       // on load miss only
 
         bool bypassL1D = false;
-        if (CACHE_GLOBAL == mf->get_inst().cache_op || (m_L1D == NULL)) {
+        if (CACHE_GLOBAL == mf->get_inst().cache_op || (m_L1D == NULL) || (mf->get_addr() >= 0xc0000000 
&& mf->get_addr() <= 0xc00fffff)) {
           bypassL1D = true;
+          m_stats->gpgpu_n_bypassL1D++;
         } else if (mf->get_access_type() == GLOBAL_ACC_R ||
                    mf->get_access_type() ==
                        GLOBAL_ACC_W) {  // global memory access
diff --git a/src/gpgpu-sim/shader.h b/src/gpgpu-sim/shader.h
index 6481790..bd69665 100644
--- a/src/gpgpu-sim/shader.h
+++ b/src/gpgpu-sim/shader.h
@@ -1689,6 +1689,8 @@ struct shader_core_stats_pod {
   unsigned *gpgpu_n_shmem_bank_access;
   long *n_simt_to_mem;  // Interconnect power stats
   long *n_mem_to_simt;
+
+  unsigned gpgpu_n_bypassL1D;
 };
 
 class shader_core_stats : public shader_core_stats_pod {
\end{verbatim}

\end{document}
