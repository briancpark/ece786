\documentclass{article}
\usepackage[utf8]{inputenc}
\usepackage[margin=0.70in]{geometry}
\usepackage{hyperref}
\usepackage{graphicx}
\usepackage{authblk}
\usepackage{subfig}
\usepackage{indentfirst}
\usepackage{minted}

\title{Program Assignment 1: CUDA Programming and GPGPU-Sim}
\author{Brian Park}
\affil{North Carolina State University, Computer Engineering 786}
\date{February 2023}

\begin{document}

\maketitle

\section{Part A: CUDA Programming}
% -For part A, explain your CUDA kernel function and report the timing differences with different memory management method.

Here we do matrix matrix multiplication of the following equation in CUDA:

Defined as:
$$\left(\begin{array}{l}
a_{b_{n-1}, \ldots, b_{t+1}, 0, b_{t-1}, \ldots, b_0}^{\prime} \\
a_{b_{n-1}^{\prime}, \ldots, b_{t+1}, 1, b_{t-1}, \ldots, b_0}
\end{array}\right)=\left[\begin{array}{ll}
U_{0,0} & U_{0,1} \\
U_{1,0} & U_{1,1}
\end{array}\right]\left(\begin{array}{l}
a_{b_{n-1}, \ldots, b_{t+1}, 0, b_{t-1}, \ldots, b_0} \\
a_{b_{n-1}, \ldots, b_{t+1}, 1, b_{t-1}, \ldots, b_0}
\end{array}\right)$$

The serial code for CPU is as follows:
\begin{minted}{C}[H]
void quantum_simulation_cpu(float* U, float* a, float* output, size_t qubit, size_t N) {
    // Perform quantum simulation on qubit
    for (size_t i = 0; i < N; i++) {
        if ((i & (1 << qubit)) == 0) {
            output[i] = U[0] * a[i] + U[1] * a[i + (1 << qubit)];
        } else {
            output[i] = U[2] * a[i - (1 << qubit)] + U[3] * a[i];
        }
    }
}    
\end{minted}

And this is the transformed CUDA code:
\begin{minted}{Cuda}[H]
__global__ void quantum_simulation_gpu(const float* U, const float* a, float* output, int qubit,
                                       int N) {
    int tid = blockDim.x * blockIdx.x + threadIdx.x;

    register size_t qid = 1 << qubit;

    if (tid > N)
        return;

    if (tid & qid)
        output[tid] = U[2] * a[tid - qid] + U[3] * a[tid];
    else
        output[tid] = U[0] * a[tid] + U[1] * a[tid + qid];
}
\end{minted}

I divide the work in the kernel simply by observing patterns in the formula. I notice that per every $1 << qubit$, there are contiguous dot products done on the elements. By simply doing boolean algebra, I can filter out the indices to perform dot products on. I can add an if else statement to make sure that elements don't collide when doing computations, as the first $a$ 0s out the $a_{qubit}$ and the second $a$ fills out $a_{qubit}$ with 1. Note that this will probably incur branch divergence due to conditional statement. Some further optimization sare done such as pinning qid to a register. We could replace the dot product operations with FMADD (fused multiply add) to double performance. Since the compiler flag is already set to \verb|-O3|, observing the PTX assembly using \verb|nvcc -ptx -O3 quamsimV1.cu| we see that compiler already handled that for us by using \verb|fma.rn.f32| instruction.

\subsection{Benchmarks}
The physical hardware used for the GPUs were the Hydra clusters, which are equipped with a mix of NVIDIA TITAN V and NVIDIA TITAN Xp.

I benchmarked the CUDA kernel for 100 trials on the NVIDIA TITAN V GPU.

On \verb|input.txt| the execution time for version 1 is 5.532 $\mu s$ for the quantum simulation kernel using \verb|cudaMalloc|. For version 2 using CUDA unified memory semantics via \verb|cudaMallocManaged|, the speed is 10.010 $\mu s$, which is $1.81 \times$ slower than version 1.

The performance can also be defined not only in latency, but also FLOPS (floating point operations per second). Thus, the performance for version 1 is 69.42 GFLOPS and for version 2 is 38.36 GFLOPS. We see that the second version really takes a hit on the execution time with a tradeoff of programmability. Explicitly managing memory between the CPU and GPU has substantial benefits over letting CUDA handle it via their unified memory programming model.  

\section{Part B: GPGPU-Sim}
% -For part B, look at the specification of part B, finish the part of "Compile your program and run your program with GPGPU-SIM" and answer the questions which are described in step 9.

% Run your program with the performance simulator (modify the file of gpgpusim.config to change the option -gpgpu_ptx_sim_mode to 0 ), and try to analysis the following statistics with the sampleInput:

% i. What is the IPC of your program and how is this value calculated from the statistics? 
% ii. What is the data cache miss_rate and how is this value calculated from the statistics? 
% iii. You should be able to find these statistics directly from the simulator output.

When running the quantum simulation code using GPGPU simulator the IPC of the program is 1.29 instructions per cycle. This is calculated using the \verb|gpgpu_simulation_rate| instructions per second and cycles per second reported, which were 261 (inst/sec) and 202 (cycle/sec) respectively.

The data cache miss rate is 100\% for L1 and for L2, it is 48.57\%. The rates are reported via \verb|L1D_total_cache_miss_rate| and \verb|L2_total_cache_miss_rate|, but can also be calculated by dividing the reported misses by total cache accesses per each level.

\end{document}

comp