\documentclass[11pt]{article}
\usepackage{ece786}

%%%%%%%%%%%%%%%%%%%% name/id
\rfoot{\small Brian Park | 200190057}


%%%%%%%%%%%%%%%%%%%% Course/HW info
\newcommand*{\instr}{Huiyang Zhou}
\newcommand*{\term}{Spring 2023}
\newcommand*{\coursenum}{ECE 786}
\newcommand*{\coursename}{Advanced Computer Architecture: Data Parallel Processors}
\newcommand*{\hwnum}{1}

\rhead{\LARGE   \fontfamily{lmdh}\selectfont	HW \hwnum}

\lfoot{\small \coursenum, \term, HW \hwnum}

%%%%%%%%%%%%%%%%%%%%%%%%%%%%%% Document Start %%%%%%%%%%%%%%%%%
\begin{document}


%%%%%%%%%%%%%%%%%%%%%%%%%%%%%%%%%%%%%%%%%%%%%%%%%%%%%%%%%%%%%%%%%%%%%%%%%%%%%%%%%%%%%%%%
% Question 1
%%%%%%%%%%%%%%%%%%%%%%%%%%%%%%%%%%%%%%%%%%%%%%%%%%%%%%%%%%%%%%%%%%%%%%%%%%%%%%%%%%%%%%%%
\section{}

In this exercise, we analyze how the instructions of a Vector-Add kernel function run on SIMT pipelines. The following is the instruction sequence in the PTX format:

\begin{verbatim}
ld.global.f32   %f1, [%r14+0]    // f1 = mem[r14+0]
ld.global.f32   %f2, [%r16+0]    // f2 = mem[r16 + 0]
add.f32         %f3, %f1, %f2    // f3 = f1 + f2
st.global.f32   [%r18+0], %f3    // mem[r18 + 0] = f3 
\end{verbatim}

Using the pipeline timing diagram to show how these instructions are executed on one SM. The following assumptions are made: 

\begin{enumerate}
	\item the SM has 16 SPs
	\item the load/store latency is 5 cycles (i.e., 5 MEM stages) with AGEN included in the 5 stages, and the add latency is 3 cycles (i.e., 3 EX stages)
	\item the SM can run 6 warps concurrently and the warp scheduling policy is round robin
	\item there is no scoreboard to support more than 1 instruction from the same warp to be issued to the pipeline. 
	\item IF and ID are 2 cycles each
	\item WB can handle 16 threads at a time.
\end{enumerate}

How many cycles does it take to execute all the 6 warps, counting from when the first instruction enters the IF stage to when the last instruction enters the WB stage?

\newpage

\end{document}