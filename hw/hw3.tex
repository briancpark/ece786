\documentclass[11pt]{article}
\usepackage{ece786}

%%%%%%%%%%%%%%%%%%%% name/id
\rfoot{\small Brian Park | 200190057}


%%%%%%%%%%%%%%%%%%%% Course/HW info
\newcommand*{\instr}{Huiyang Zhou}
\newcommand*{\term}{Spring 2023}
\newcommand*{\coursenum}{ECE 786}
\newcommand*{\coursename}{Advanced Computer Architecture: Data Parallel Processors}
\newcommand*{\hwnum}{3}

\rhead{\LARGE   \fontfamily{lmdh}\selectfont	HW \hwnum}

\lfoot{\small \coursenum, \term, HW \hwnum}

%%%%%%%%%%%%%%%%%%%%%%%%%%%%%% Document Start %%%%%%%%%%%%%%%%%
\begin{document}


%%%%%%%%%%%%%%%%%%%%%%%%%%%%%%%%%%%%%%%%%%%%%%%%%%%%%%%%%%%%%%%%%%%%%%%%%%%%%%%%%%%%%%%%
% Question 1
%%%%%%%%%%%%%%%%%%%%%%%%%%%%%%%%%%%%%%%%%%%%%%%%%%%%%%%%%%%%%%%%%%%%%%%%%%%%%%%%%%%%%%%%
\section{}

Considering the following code in a GPU kernel function:

\begin{verbatim}
int index = blockIdx.x * blockDim.x + threadIdx.x;

if (index > x1)      // Section 1 - THEN
{   

                …  

}

else                 // Section 1 - ELSE
{

                …

}              

d_e[index] = d_c[index] + d_d[index];
\end{verbatim}

Assuming that the corresponding assembly code is as follows:

\begin{verbatim}
PC      Instruction               Comment
*0000*  MOV R1, c [0x1] [0x100];
*0020*  SSY 0xf0;                 SSY Instruction (push stack)
*0028*  ...                       ...
                                    
*0090*  @P0 BRA 0xb8;             Branch corresponding to Outer IF Then-Else 
                                  (push stack if divergent)
*0098*  LD.E R3, [R2];            ELSE PART
...
*00b0*  ST.E.S [R6], R2;          ELSE PART. Notice ".S" flag. (pop stack)
*00b8*  LD.E R5, [R4];            IF THEN PART
...     ...                       ...
*00e8*  NOP.S CC.T;               Last Instruction of Outer If-Then. Notice ".S" flag. 
                                  (pop stack)
*00f0*  LD.E R3, [R10];           Threads Synchronizes at this point.
*00f8*  LD.E R4, [R8];            ...
*0100*  LD.E R2, [R6];            Go till Exit.
...     ...                       ...
*0140*  EXIT;
\end{verbatim}
With a warp size of 8, show how the SIMT stack is updated during the execution for the following two warps. (1) The branch at \verb|PC=x0090| is taken for all 8 threads in a warp. (2) Among the 8 threads in a warp, the branch outcomes are: taken for the first 2 threads and not-taken for the remaining 6 threads. (Hint: you can refer to the slides in our course notes to see one format that we used to show the SIMT stack states)


Case (1): Branch at \verb|PC=x0090| is taken for all 8 threads in a warp (no divergence).\\

% Create a table
\begin{table}[H]
    \centering % used for centering table
    \begin{tabular}{c c c c c c} % centered columns (4 columns)
    \hline\hline %inserts double horizontal lines
    PC & Active Mask & TOS & TOS-1 & TOS-2 & Comments \\ [0.5ex] % inserts table
    %heading
    \hline % inserts single horizontal line
    0000 & \verb|11111111| & & & & \\
    0020 & \verb|11111111| & & & & Before SSY \\
    0028 & \verb|11111111| & \verb|SSY 0xf0| & & & After SSY \\
    0090 & \verb|11111111| & \verb|SSY 0xf0| & & & Before Branch \\
    00b8 & \verb|11111111| & \verb|SSY 0xf0| & & & After Branch \\
    ...  & \verb|11111111| & \verb|SSY 0xf0| & & & \\ 
    00e8 & \verb|11111111| & \verb|SSY 0xf0| & & & NOP.S (pop)\\
    00f0 & \verb|11111111| & & & & LD.E \\
    00f8 & \verb|11111111| & & & & LD.E \\
    0100 & \verb|11111111| & & & & LD.E \\
    ...  & \verb|11111111| & & & & \\
    0140 & \verb|11111111| & & & & EXIT \\

\hline %inserts single line
\end{tabular}
\label{table:nonlin} % is used to refer this table in the text
\end{table}

Case (2): Branch at \verb|PC=x0090| is taken for first two threads and not taken for other 6 (divergence)
% Create a table
\begin{table}[H]
    \centering % used for centering table
    \begin{tabular}{c c c c c c} % centered columns (4 columns)
    \hline\hline %inserts double horizontal lines
    PC & Active Mask & TOS & TOS-1 & TOS-2 & Comments \\ [0.5ex] % inserts table
    %heading
    \hline % inserts single horizontal line
    0000 & \verb|11111111| & & & & \\
    0020 & \verb|11111111| & & & & Before SSY \\
    0028 & \verb|11111111| & \verb|SSY 0xf0| & & & After SSY \\
    0090 & \verb|00111111| & \verb|SSY 0xf0| & & & Before Branch \\
    0098 & \verb|00111111| & \verb|DIV 0xf0 0xb8| & \verb|SSY 0xf0| & & After Branch \\
    ...  & \verb|00111111| & \verb|DIV 0xf0 0xb8| & \verb|SSY 0xf0| & & \\
    00b0 & \verb|00111111| & \verb|DIV 0xf0 0xb8| & \verb|SSY 0xf0| & & Before ST.E.S (pop)\\
    00b8 & \verb|11000000| & \verb|SSY 0xf0| & & & After ST.E.S \\
    ...  & \verb|11000000| & \verb|SSY 0xf0| & & & \\
    00e8 & \verb|11000000| & \verb|SSY 0xf0| & & & NOP.S (pop)\\
    00f0 & \verb|11111111| & & & & LD.E \\
    00f8 & \verb|11111111| & & & & LD.E \\
    0100 & \verb|11111111| & & & & LD.E \\
    ...  & \verb|11111111| & & & & \\
    0140 & \verb|11111111| & & & & EXIT \\
\hline %inserts single line
\end{tabular}
\label{table:nonlin} % is used to refer this table in the text
\end{table}
\newpage

\end{document}